\section{Project Scope}\label{project-scope}

\subsection{Aims and objectives}\label{aims-and-objectives}

What I will be developing over the upcoming months is an app to help
users get out of bed easier in the morning in a useful and information
rich way.

My aims are:

\begin{itemize}
\tightlist
\item
  Produce an alarm app with all the functionality users are used to.
\item
  Integrate Smartbulb functionality into the app to turn the light on in
  the morning with the alarm.
\item
  To turn off the lights at night without having to get out of bed.
\item
  Provide weather information for the day.
\item
  Inform the user of their schedule for the day and upcoming events.
\item
  Publish the application to the play store for download and use by
  others.
\end{itemize}

\subsection{Stakeholders}\label{stakeholders}

I do not have a client that I am developing my application for and so do
not have a pre-defined user base or stakeholders however, I have
identified the following stakeholders:

\begin{itemize}
\item
  Myself - Not only am I developing the application making me a
  stakeholder, I am also very interested in home automation and waking
  up happy.
\item
  My supervisor, Marcus Winter - By accepting to be my supervisor Marcus
  is also a stakeholder for my application, he will be providing
  feedback and assistance through out development and will ultimately be
  grading me on my efforts.
\item
  An expanding user base of smartbulbs - Although the market currently
  is small the cost of smartbulbs is decreasing making them more
  available to users.
\item
  Anyone that uses an alarm - The largest stakeholder I have is anyone
  that uses an alarm, many use the alarm that comes on their phone and
  others use stand-alone alarm clocks. By investigating the most popular
  alarms used by people I will be able to get vital information on what
  makes for a good alarm and what I should avoid.
\end{itemize}

\subsection{Communications}\label{communications}

I will maintain contact with my project supervisor with monthly meetings
where I intend to measure my progress against deadlines and goals,
reflect on what progress has been made and address issues, challenges
and for development advice to assist me successfully complete my project
as planned.

Regular emails will also be used between meetings to keep in contact and
keep my supervisor informed of what I intend to talk about and on my
progress made.

\subsection{Installation Process}\label{installation-process}

By developing for Android I will be able to make the app installation
process very simple by publishing it to the Play store. To install an
app from the Play store you can simply select the option to install the
app and it will be downloaded and installed seamlessly.

I will be ensuring to develop to a high standard and ensure there are no
issues, bugs or flaws with my application. By developing for the KitKat
version I will be able to support over an 80\% market share.
\cite{androidversion}

\subsection{Quality checks}\label{quality-checks}

During development I will ensure to maintain my code and follow the
principles that have been taught to me and that I have learnt and will
learn, in doing so my code should be easily maintainable, readable and
extendable for possible extensions and stretch goals.

I will develop a test plan as I continue to develop my application to
allow me to note issues and ensure previous functionality has not been
effected by further developments.

\subsection{How will I measure
success?}\label{how-will-i-measure-success}

My key performance indicators are outlined below:

\begin{itemize}
\tightlist
\item
  Alarm functionality
\item
  Smartbulb integration
\item
  Weather functionality
\item
  Calendar Integration (Stretch)
\item
  Text to speech (Stretch)
\end{itemize}

If I am unable to produce a working alarm app with smartbulb
functionality I will have failed to achieve what I intended to develop
and so these are my highest priority.
