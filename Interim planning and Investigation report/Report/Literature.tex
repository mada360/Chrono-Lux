\section{Literature Review}\label{literature-review}

There a currently several home automation systems available to the
public market including smart heating with the \cite{hive} and
\cite{nest}, automated lighting with the \cite{philipshue} and
integrated automation in the form of the \cite{echo} which provides a
wide array of functionality and can be linked with many of the
previously mentioned systems to provide whole home automation.

The market of home automation is growing while the cost of entry
decreases, with advancements in LED technology the \cite{philipshue} has
become more affordable. The \cite{echo} is also at a very affordable
price with Amazon keeping the costs low to increase market saturation.

Current devices use the Bluetooth low energy and Wi-Fi standards, often
with a internet connected hub to handle communications. Wi-Fi is
relatively power intensive and provides a bandwidth excessively large
for the application, while BLE has limited range.

New standards \cite{zwave} and \cite{zigbee} are being developed, both
using far less power allowing for extended battery life. Both have
limited range like BLE however \cite{zwave} is designed to create an
interconnected network between devices to maintain low power while
extending the range.

\subsection{Philips Hue}\label{philips-hue}

The \cite{philipshue} system uses the \cite{zigbee} standard, though all
of this is transparent as the method of interfacing with the devices is
to use `GET', `PUSH', `POST' and `PUT' URL requests and provide JSON
formatted commands in the body to interact.

The state of a specific light can be received using `GET' and providing
the URL /api/devID/lights/1 or all of the lights by not specifying the
number.

The state can be changed using `PUT' instead and providing attributes
and their values that you would like to change, for example:

\begin{lstlisting}
{"on":true, "bri":255} 
\end{lstlisting}

A few useful attributes: on = true/false bri = Brightness between 0 and
254

Colour settings include: sat = Saturation between 0 and 254 hue = The
hue of the light (hue runs from 0 to 65535)

\subsection{Human Perception of Light}\label{human-perception-of-light}

Many human senses are based on a logarithmic scale, this is to say we
are far more able to distinguish changes in light or sound in the lower
band of the senses compared to higher, so whispering occurring will have
a more distinguishable change in volume than two jet engines roaring.

The same applies to sight, it is more important to distinguish details
in low light such as that from the moon compared to the light change of
daylight at varying times of the day. We do this to normalise our senses
to best suit our environment.

This kind of stimulus perception is defined as the just-noticeable
difference (JND). Firs summarised by Ernst Weber in 1834 his equation
was called Weber's Law and simply stated that response intensity
increases as stimulus intensity increases
\parencite[p. 1613-1615]{salkind2010encyclopedia}. Further refined by
Gustav Fechner who proposed the use a constant to provide a curve to the
stimulus/perception relationship. Fechners' law was a much better fit,
however some stimulus did not fit well, such as that of electric shock.

Most recently in the 60's an American psychologist S. S. Stevens
produced a formula that worked for all forms of stimulus, even for
electric shocks \parencite{stevens1957psychophysical}. He proposed an
exponential function raising the data to a power rather than using a
simple constant. This essentially stated that to get a linear increase
in perception of various stimuli, the stimulus would need to increase in
an exponential form.
