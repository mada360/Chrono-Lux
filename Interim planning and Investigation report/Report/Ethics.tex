\section{Ethical Considerations}\label{ethical-considerations}

\subsection{Ethical Q\&A}\label{ethical-qa}

\subsubsection{What are the aims of the
research?}\label{what-are-the-aims-of-the-research}

The aim of my research will to obtain user feedback of the usage of my
app.

\subsubsection{What are the methods of data collection and
analysis?}\label{what-are-the-methods-of-data-collection-and-analysis}

I will provide the users a device with my app installed on it and ask
them to perform various tasks and rate them in how easy they found the
task on a scale of 1-10.

Another means of feedback will be for design feedback and for how they
feel the design could be improved.

\subsubsection{Who are your
participants?}\label{who-are-your-participants}

My participants will be my peers and family, all of which are over the
age for informed consent and fall between the ages of 20 and 60.

\subsubsection{How will they be selected and
recruited?}\label{how-will-they-be-selected-and-recruited}

There will be no means for selection as I intend to keep the research
fairly limited to allow me to quickly review the feedback and allow me
to make improvements based on the feedback.

\subsubsection{How will the sample size be
determined?}\label{how-will-the-sample-size-be-determined}

The sample size I intend to be at least 10 users, however due to the
set-up required for my app it will be quite difficult to perform
research on a larger group. This is due to the network connectivity
required.

\subsubsection{What will your participants have to
do?}\label{what-will-your-participants-have-to-do}

As sated above, the users will be asked to perform several simple and
more complex tasks within my app so I can obtain feedback, this will
include actions such as turning an alarm on and off, switching the
lighting on and off and navigating the settings menu.

\subsubsection{What potential risks to the participants do you
anticipate?}\label{what-potential-risks-to-the-participants-do-you-anticipate}

There should be no risk to the participants as they will be seated
during the testing and I expect the entire test to last no more than 10
minutes.

\subsubsection{How will you minimise/eliminate these
risks?}\label{how-will-you-minimiseeliminate-these-risks}

Due to the minimal risk there will be no need to minimise or eliminate
any.

\subsubsection{What potential risk to the researcher do you
anticipate?}\label{what-potential-risk-to-the-researcher-do-you-anticipate}

As with the participant the researcher should have no risks posed on
them.

\subsubsection{How will you brief and debrief your
participants?}\label{how-will-you-brief-and-debrief-your-participants}

I will brief the participants by outlining the test I would like them to
perform. The debrief will consist of question on any extra feedback not
already provided and on how I could improve testing for potential future
user testing.

\subsubsection{Will informed consent be
sought?}\label{will-informed-consent-be-sought}

It is important for me to obtain informed consent and I will ensure that
the participants are fully aware of what it is I will ask of them. All
participants will be over 18 and have no learning difficulties and so
will be able to provide full informed consent.

\subsubsection{If subjects are unable to give informed consent, what
steps have you taken to ensure they are willing to
participate?}\label{if-subjects-are-unable-to-give-informed-consent-what-steps-have-you-taken-to-ensure-they-are-willing-to-participate}

This is not an issue as all participants will be able to provide full
consent.

\subsubsection{Will participants be able to withdraw without
penalty?}\label{will-participants-be-able-to-withdraw-without-penalty}

A participant will be able to withdraw at any time and there will be no
repercussion for them doing so. Any partial research will be used up to
the point of with drawl unless they do not want it to be, in which case
the participants research will be destroyed.

\subsubsection{How do you propose to ensure participant's
confidentiality and
anonymity?}\label{how-do-you-propose-to-ensure-participants-confidentiality-and-anonymity}

The testing will be conducted with the participant and myself
supervising. I will be present but not providing assistance with the
usage of the app and only there to help with other issues or to full
fill the data destruction provided the user would like to withdraw.

All data obtained will be written down on a questionnaire of which the
user will not need to write their name or other identifying details. The
papers will also be shuffled prior to processing the data to further
obscure the participant feedback.

\subsubsection{How and where will the data be
stored?}\label{how-and-where-will-the-data-be-stored}

The data will be stored formally on physical questionnaires. The data
will then be processed into a more usable format such as an Excel
spreadsheet and stored on my password protected devices and accounts and
will not be shared with any third party.
