\section{Success and Failure}\label{success-and-failure}

In this section is an assessment of some of the successes or failures of
the project as a whole.

\subsection{Successes}\label{successes}

Within the project there were some highlights of success that are
mentioned as follows.

\subsubsection{The Project}\label{the-project}

As outlined within the project specification several goals and stretch
goals were set and if the goals were meant this would regard the project
as being a success. The application provides; alarm functionality, smart
bulb integration and weather functionality each of which are the main
goals set and as such the project was a success.

\subsubsection{Using SQLite}\label{using-sqlite}

I had not foreseen the need to use a database within my application as
the storage of alarms had not seemed to be a complex enough scenario to
need a database. As the other storage methods available were not suited
for storing an object with multiple variables however it was necessary
to utilise the SQLite database for storage.

Though more complex than other storage methods, the use of database
helpers reduced the need to write raw SQL queries and requests and
allowed for a more programmatic style to interface with the database.
Overall the database worked well.

\subsubsection{Pending intents}\label{pending-intents}

When developing the alarm and using the alarm manager it was necessary
to use pending intents. Pending intents are simple enough to understand
however their implementation is a little odd, as cancelling a pending
intent requires another matching pending intent being produced, however
the use of the \lstinline!cancel()! method at the end cancels the
pending intent.

Due to this and the need to allow the user to enable and disable an
alarm, it was necessary to leave the pending intent and instead to check
the alarm that was being triggered was enabled. In doing so there is no
longer a need to create and delete pending intents for alarms that had
been disabled, as the alarm object would still exist within the database
and could cause errors if there was an ID mismatch.

\subsection{Failures}\label{failures}

Although the project was successful, there are several aspects that
failed to be implemented or did not make it into the final application.

\subsubsection{No Adjustable alarm}\label{no-adjustable-alarm}

Due to the time picker not returning values, the alarms are created from
within the listener method, because of this it made it difficult to be
able to allow for alarms to be changed once set. The lack of ability to
pass values into the time picker means that there is no way to pass an
alarm id into the method either, something that could allow for a check
to see if an alarm by that ID already exists, if the alarm does exist
adjust the time and if not create a new alarm.

\subsubsection{Not Reaching Stretch
Goals}\label{not-reaching-stretch-goals}

This could be classed as not being a failure, as the main goals outlined
have been met. However I feel that the stretch goals would have worked
well within my application and helped it to stand out more through being
more than a regular alarm application, more so than the inclusion of
controlling smart-lights.
