\section{Stages of Development}\label{stages-of-development}

For a complete project there are fundamental stages that need to be
addressed, failure to address these stages can result in project drift
resulting in the end result not fulfilling expectations or failing to
achieve the requirements of the project.

It must be mentioned that although the term stages has been used, in
many methodologies especially those that involve rapid development many
of these stages will occur in either parallel or in quick succession;
for example testing will often happen throughout development to ensure
aspects of the code and design work as expected before extending the
software further.

\subsection{Analysis}\label{analysis}

The first stage of development is analysis, failing to analyse the
problem or obtain customer requirements renders development practically
impossible as the project scope can not be defined. Analysis often
requires the assessment of the problem or product to be developed and
rendering it down into more fundamental pieces such as how to implement
a RESTful API or store any data required. During this stage it is common
to perform competitor analysis by finding similar existing solutions to
challenges that may be faced and to implement improvements over the
competition and produce something objectively better to improve
usability, experience, performance and any combination of aspects that
can be deemed desirable.

\subsubsection{Problem Analysis}\label{problem-analysis}

\subsubsection{Competitors}\label{competitors}

\paragraph{\texorpdfstring{\href{http://sleep.urbandroid.org/}{Sleep as
Android}}{Sleep as Android}}\label{sleep-as-android}

This is the most feature packed alarm app available on Android that I
could find.

\paragraph{\texorpdfstring{\href{https://cuckuu.com/}{Cuckuu}}{Cuckuu}}\label{cuckuu}

Cuckuu doesn't have the same level of integration with reminders,
appointments or weather as what I intend and it doesn't include any
smart bulb integration.

\subparagraph{Appearance}\label{appearance}

\subparagraph{Features}\label{features}

\paragraph{\texorpdfstring{\href{https://wakie.com/}{Wakie}}{Wakie}}\label{wakie}

Wakie is very different in how it intends to wake a user up and besides
being an alarm has little to what my app will consist of. Wakie does
have a very interesting perms though, that a stranger from around the
world will be able to call you when you would like to be woken and talk
about a topic you would like to discuss.

\paragraph{Appearance}\label{appearance-1}

With a very simple and clean design this app definitely looks nice and
allows itself to easily be ported into the varying mobile platforms.

\paragraph{Features}\label{features-1}

\subsection{Design}\label{design}

During the design stage the User Interface (UI) and various aspects of
the code are planned for development. By critically thinking about
certain design aspects with regards to the requirements outlined from
the analysis stage, designs of the software can be produced, this
includes both the visual designs of the user interface and some of the
software solutions that may be required such as certain tooling and
frameworks that are available for development to produce more stable
software faster by removing the need to produce custom solutions to
aspects that have been solved and are found throughout programming.

\subsubsection{Designs}\label{designs}

APPENDIX?

\subsection{Development}\label{development}

Throughout development everything that has been produced from both the
analysis and design stages are implemented through creation and
utilisation of software and assets such as images and sounds to produce
the goal of the project, in the instance of a mobile application this
would consist of producing and application that can run on the required
platforms such as Android, iOS or Windows Phone and for the application
to perform the tasks required such as sending and receiving messages and
displaying them to the user in the case of a communications application.

\subsubsection{Platform}\label{platform}

\subsubsection{Target Market}\label{target-market}

\subsection{Testing}\label{testing}

Often testing is an ongoing aspect of the development life cycle so that
as aspects of the application are developed tests and unit testing
classes are produced to allow the automation of testing. Performing
ongoing testing is useful to indicate that code is working correctly and
if during development if the automated testing indicates any issues
these can be resolved as they occur instead of during feedback or after
deployment, both of which increase development cost through having to
hunt down bugs and maintain the application during the life time of the
application.

\subsubsection{Tests}\label{tests}

\subsection{Feedback}\label{feedback}

Feedback is crucial for assessing the success of a project, by obtaining
feedback from testers or trial users changes can be made to improve the
experience of the application. Feedback can be obtained during
development by having people use certain aspects of the application as
it's being developed. If during feedback multiple users raise complaints
about the same aspects this is a key indicator to change that aspect of
the application before full deployment to ensure a higher level of
polish and maintain a higher level of respect and image.

\subsubsection{User Trials}\label{user-trials}
