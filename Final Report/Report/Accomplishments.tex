\section{Accomplishments}\label{accomplishments}

The following are some of the accomplishments within developing my
application of which I am most proud.

\subsection{The Ability to Control
Lighting}\label{the-ability-to-control-lighting}

There are very applications available that allow for the connection to
and manipulation of smart-lights from within the application. Because of
my app being one of these few I feel is an accomplishment.

\subsection{The Application UI Design}\label{the-application-ui-design}

Overall I feel the look and feel of application works very well; it
adheres to the Android design guidelines \parencite{androiddesign} and
feels like a native application by maintaining a cohesive user
experience.

\subsubsection{Tab Navigation}\label{tab-navigation}

The tab navigation works well, providing the user with a simple and
clean way for using my application. By providing only the relevant
functionality for each tab the user experience is simple and intuitive.
Through avoiding too much being displayed on the screen at once,
relevant information can be seen instantly ensuring ease of use and
allowing the user to only spend as long as necessary within the
application.

Through including tab navigation it was required to use fragments within
my application, fragments are not as simple to use as a regular activity
and pose multiple design challenges over an application activity. The
extra difficulty associated with fragments did slow the initial
development of my application and my initial tab view needed to be
replaced due to the use of several deprecated styles. Despite this
challenge I feel the fragments work well within my application and the
functionality I outlined within my project has been accomplished.

\subsubsection{Snackbars and Toasts}\label{snackbars-and-toasts}

By developing for newer versions of Android I utilised the Snackbar
notification system introduced in API level 23. The snackbar in most
instances replaces the original Toast notification system. Snackbars
improve upon toast in many ways as they; provide information to the user
and allow the user to interact with the notification by performing an
action if provided by the snackbar, such as undoing an action that
triggered the snackbar to be displayed. The snackbar can also be
dismissed by swiping, while the toast is displayed until the specified
display time has passed.

Toasts are stilled used within the newer versions of Android, however
they are now often used for system notification as they can be displayed
without being associated with an activity and as they can't be dismissed
are good for showing warnings or important information. Snackbars also
blend better within an application, showing up above or moving elements
such as a floating action button, ensuring that the notification can be
seen and that aspects of the view are not obstructed.

It is specified within the guidelines that a snackbar should be
displayed as long as necessary, as such short notifications should not
be displayed for a long time, while notifications that provide an action
or consist of a lot of text should be displayed for longer to allow the
user to fully read the text or provide adequate time for interaction.
