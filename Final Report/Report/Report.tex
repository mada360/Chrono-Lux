%%%%%%%%%%%%%%%%%%%%%%%%%%%%%%%%%%%%%%%%%
% University Assignment Title Page
% LaTeX Template
% Version 1.0 (27/12/12)
%
% This template has been downloaded from:
% http://www.LaTeXTemplates.com
%
% Original author:
% WikiBooks (http://en.wikibooks.org/wiki/LaTeX/Title_Creation)
%
% License:
% CC BY-NC-SA 3.0 (http://creativecommons.org/licenses/by-nc-sa/3.0/)
%
% Instructions for using this template:
% This title page is capable of being compiled as is. This is not useful for
% including it in another document. To do this, you have two options:
%
% 1) Copy/paste everything between \begin{document} and \end{document}
% starting at \begin{titlepage} and paste this into another LaTeX file where you
% want your title page.
% OR
% 2) Remove everything outside the \begin{titlepage} and \end{titlepage} and
% move this file to the same directory as the LaTeX file you wish to add it to.
% Then add \input{./title_page_1.tex} to your LaTeX file where you want your
% title page.
%
%%%%%%%%%%%%%%%%%%%%%%%%%%%%%%%%%%%%%%%%%

%----------------------------------------------------------------------------------------
%	PACKAGES AND OTHER DOCUMENT CONFIGURATIONS
%----------------------------------------------------------------------------------------

%Fixes pandoc tightlist error
\providecommand{\tightlist}{%
  \setlength{\itemsep}{0pt}\setlength{\parskip}{0pt}}

\documentclass[12pt]{article}
%\renewcommand{\familydefault}{\sfdefault}
\usepackage{amsfonts}
\usepackage{amssymb}
\usepackage{multicol}
\usepackage{graphicx}
\usepackage{subfig}
\usepackage{pdfpages}

\usepackage{titlesec}
\usepackage{longtable}
\usepackage{supertabular}

\usepackage{booktabs}
\setlength{\parindent}{0pt}
\setlength{\parskip}{1em}

\usepackage{hyperref}
\usepackage{pdflscape}

% Contents of listings-setup.tex
\usepackage{framed, xcolor}
\usepackage{listings}
\usepackage{framed}

\lstset{
    basicstyle=\ttfamily,
    numbers=left,
    keywordstyle=\color[rgb]{0.13,0.29,0.53}\bfseries,
    stringstyle=\color[rgb]{0.31,0.60,0.02},
    commentstyle=\color[rgb]{0.56,0.35,0.01}\itshape,
    numberstyle=\footnotesize,
    stepnumber=1,
    numbersep=5pt,
    backgroundcolor=\color[RGB]{248,248,248},
    showspaces=false,
    showstringspaces=false,
    showtabs=false,
    tabsize=2,
    captionpos=b,
    breaklines=true,
    breakatwhitespace=true,
    breakautoindent=true,
    escapeinside={\%*}{*)},
    linewidth=\textwidth,
    basewidth=0.5em,
}

%References

\usepackage[
backend=bibtex,
style=numeric,
autocite=footnote,
citestyle=authoryear
]{biblatex}

\addbibresource{references.bib}



\usepackage[british]{babel}
\usepackage{csquotes}

\usepackage{url}

\usepackage{microtype}

\usepackage{float}


%	Below are optional commands to adjust the document.
%----------------------------------------------------------------------------------------
%----------------------------------------------------------------------------------------
%	Show's overfilled areas.
%----------------------------------------------------------------------------------------
%\overfullrule=2cm
%----------------------------------------------------------------------------------------

%-------------------------'s you---------------------------------------------------------------
%	Stop hyphenation.
%----------------------------------------------------------------------------------------
%\widowpenalty=10000
%\clubpenalty=10000
%----------------------------------------------------------------------------------------

%----------------------------------------------------------------------------------------
%	Adds a clear page after each section to start the next on a new page.
%----------------------------------------------------------------------------------------
%\newcommand\sectionbreak{\clearpage}
%----------------------------------------------------------------------------------------

%----------------------------------------------------------------------------------------
%	The following removes section numbering
%----------------------------------------------------------------------------------------
%\renewcommand{\thesection}{}
%\renewcommand{\thesubsection}{\arabic{section}.\arabic{subsection}}
%\makeatletter
%\def\@seccntformat#1{\csname #1ignore\expandafter\endcsname\csname the#1\endcsname\quad}
%\let\sectionignore\@gobbletwo
%\let\latex@numberline\numberline
%\def\numberline#1{\if\relax#1\relax\else\latex@numberline{#1}\fi}
%\makeatother
%----------------------------------------------------------------------------------------

\begin{document}

\begin{titlepage}

\newcommand{\HRule}{\rule{\linewidth}{0.5mm}} % Defines a new command for the horizontal lines, change thickness here

\center % Center everything on the page

%----------------------------------------------------------------------------------------
%	HEADING SECTIONS
%----------------------------------------------------------------------------------------

\textsc{\LARGE University of Brighton}\\[1.5cm] % Name of your university/college
\textsc{\Large Computer Science (Games) BSc(Hons)}\\[0.5cm] % Major heading such as course name
\textsc{\large Individual Project - CI301}\\[0.5cm] % Minor heading such as course title

%----------------------------------------------------------------------------------------
%	TITLE SECTION
%----------------------------------------------------------------------------------------

\HRule \\[0.4cm]
{ \huge \bfseries Final Year Project Report}\\[0.4cm] % Title of your document
\HRule \\[1.5cm]

%----------------------------------------------------------------------------------------
%	AUTHOR SECTION
%----------------------------------------------------------------------------------------

\begin{minipage}{0.4\textwidth}
\begin{flushleft} \large
\emph{Author:}\\
Adam \textsc{Worley}\\ % Your name
\emph{Student Number:}\\
13842206
\end{flushleft}
\end{minipage}
~
\begin{minipage}{0.4\textwidth}
\begin{flushright} \large
\emph{Supervisor:} \\
Marcus \textsc{Winter} % Supervisor's Name (Manos)
\end{flushright}
\end{minipage}\\[3cm]
% If you don't want a supervisor, uncomment the two lines below and remove the section above
%\Large \emph{Author:}\\
%John \textsc{Smith}\\[2cm] % Your name

%----------------------------------------------------------------------------------------
%	DATE SECTION
%----------------------------------------------------------------------------------------

{\large Hand in Date : 11\textsuperscript{th} of May 2017}\\[2cm] % Date, change the \today to a set date if you want to be precise

%----------------------------------------------------------------------------------------
%	LOGO SECTION
%----------------------------------------------------------------------------------------

\includegraphics[scale=0.60]{Images/BrightonLogo.jpg}\\[1cm] % Include a department/university logo - this will require the graphicx package

%\autocite{brightonlogo}


%----------------------------------------------------------------------------------------

\vfill % Fill the rest of the page with whitespace

\end{titlepage}

\tableofcontents
\pagebreak

\section{Methodology}\label{methodology}

A methodology is a set of methods, rules and restrictions combined to
form a procedure or discipline.

By adhering to a development methodology it is possible to set
development goals and a way to identify trouble areas during a project
allowing for plans and contingencies to be allocated to improve the
likelihood of success.

\subsection{Overview of types of
methodology}\label{overview-of-types-of-methodology}

There are many forms of methodologies, below are a few of the most
popular and an overview of each.

\subsubsection{Rapid Applications Development
(RAD)}\label{rapid-applications-development-rad}

By producing prototypes of the software quickly customers are able to
test and provide feedback as the software is developed. This is useful
as often requirements change and it's common for developers to produce
software that isn't actually what the customer wanted.

\subsubsection{Agile}\label{agile}

Originally project management was slow to adapt to changes with user
review coming in late stages of development. Agile however aims for
incremental development with regular feedback. \parencite{agile}

The most popular form of agile development is the Scrum
\parencite{agile} scrum is suited towards small teams and requires close
involvement by the product owner to provide regular feedback and review.

\subsubsection{Lean}\label{lean}

Much like scrum and other agile methodologies, lean aims to produce
software quickly and involves close coordination with the product owner,
lean differs by reducing waste by selecting the most valuable features
required.\parencite{agilemethods}.

\subsubsection{Waterfall}\label{waterfall}

Focuses on phases such as; requirement gathering, analyses, development
and testing. Each phase is completed entirely before moving onto the
next phase and is often depicted by the phases flowing steadily
downwards resembling a waterfall.

\subsubsection{Spiral}\label{spiral}

The spiral model is based on the incremental model and consists of four
phases; Planning, risk analysis, engineering and evaluation
\parencite{spiral}. A project will go through each phase multiple times
in an iterative process or spirals. This is very well illustrated in
figure \ref{fig:spiral}.

\begin{figure}
\centering
\includegraphics{../../Images/Spiral-model.jpg}
\caption{Spiral model diagram \parencite{spiral} \label{fig:spiral}}
\end{figure}

\subsubsection{Time Boxing}\label{time-boxing}

Involves strict deadlines rather than goals. By developing up to the
agreed upon time and evaluating progress this can allow for steadier
progress and a set time in mind which provides an achievable deadline.

Evaluating at the end of the time frame can show struggles in the
development process and provides the ability to address them rather than
simply spending more time to complete the goal.

\subsection{Choice of methodology}\label{choice-of-methodology}

After assessing the various forms of project methodologies I have
decided to use an agile methodology most notably the Lean methodology as
this will provide me the ability to develop core functionality at a fast
pace and add other features time permitting. To assist my development I
will also be using time boxing to allocate time for my applications
functions and allow me to perform regular performance reviews so I can
identify time sinks and other issues to allow me to manage them.

\subsection{Project Time line}\label{project-time-line}

Below is a Gantt chart of the overall plan for my project. A Gantt chart
doesn't suit my development methodology very well and so is fairly
high-level overview.

\begin{landscape}
\begin{figure}[htbp]
\includegraphics{Images/gantt.png}
\caption{Project Gantt Chart}
\end{figure}
\end{landscape}

\section{Stages of Development}\label{stages-of-development}

For a complete project there are fundamental stages that need to be
addressed, failure to address these stages can result in project drift
resulting in the end result not fulfilling expectations or failing to
achieve the requirements of the project.

It must be mentioned that although the term stages has been used, in
many methodologies especially those that involve rapid development many
of these stages will occur in either parallel or in quick succession;
for example testing will often happen throughout development to ensure
aspects of the code and design work as expected before extending the
software further.

\subsection{Analysis}\label{analysis}

The first stage of development is analysis, failing to analyse the
problem or obtain customer requirements renders development practically
impossible as the project scope can not be defined. Analysis often
requires the assessment of the problem or product to be developed and
rendering it down into more fundamental pieces such as how to implement
a RESTful API or store any data required. During this stage it is common
to perform competitor analysis by finding similar existing solutions to
challenges that may be faced and to implement improvements over the
competition and produce something objectively better to improve
usability, experience, performance and any combination of aspects that
can be deemed desirable.

\subsubsection{Problem Analysis}\label{problem-analysis}

\subsubsection{Competitors}\label{competitors}

\paragraph{\texorpdfstring{\href{http://sleep.urbandroid.org/}{Sleep as
Android}}{Sleep as Android}}\label{sleep-as-android}

This is the most feature packed alarm app available on Android that I
could find.

\paragraph{\texorpdfstring{\href{https://cuckuu.com/}{Cuckuu}}{Cuckuu}}\label{cuckuu}

Cuckuu doesn't have the same level of integration with reminders,
appointments or weather as what I intend and it doesn't include any
smart bulb integration.

\subparagraph{Appearance}\label{appearance}

\subparagraph{Features}\label{features}

\paragraph{\texorpdfstring{\href{https://wakie.com/}{Wakie}}{Wakie}}\label{wakie}

Wakie is very different in how it intends to wake a user up and besides
being an alarm has little to what my app will consist of. Wakie does
have a very interesting perms though, that a stranger from around the
world will be able to call you when you would like to be woken and talk
about a topic you would like to discuss.

\paragraph{Appearance}\label{appearance-1}

With a very simple and clean design this app definitely looks nice and
allows itself to easily be ported into the varying mobile platforms.

\paragraph{Features}\label{features-1}

\subsection{Design}\label{design}

During the design stage the User Interface (UI) and various aspects of
the code are planned for development. By critically thinking about
certain design aspects with regards to the requirements outlined from
the analysis stage, designs of the software can be produced, this
includes both the visual designs of the user interface and some of the
software solutions that may be required such as certain tooling and
frameworks that are available for development to produce more stable
software faster by removing the need to produce custom solutions to
aspects that have been solved and are found throughout programming.

\subsubsection{Designs}\label{designs}

APPENDIX?

\subsection{Development}\label{development}

Throughout development everything that has been produced from both the
analysis and design stages are implemented through creation and
utilisation of software and assets such as images and sounds to produce
the goal of the project, in the instance of a mobile application this
would consist of producing and application that can run on the required
platforms such as Android, iOS or Windows Phone and for the application
to perform the tasks required such as sending and receiving messages and
displaying them to the user in the case of a communications application.

\subsubsection{Platform}\label{platform}

\subsubsection{Target Market}\label{target-market}

\subsection{Testing}\label{testing}

Often testing is an ongoing aspect of the development life cycle so that
as aspects of the application are developed tests and unit testing
classes are produced to allow the automation of testing. Performing
ongoing testing is useful to indicate that code is working correctly and
if during development if the automated testing indicates any issues
these can be resolved as they occur instead of during feedback or after
deployment, both of which increase development cost through having to
hunt down bugs and maintain the application during the life time of the
application.

\subsubsection{Tests}\label{tests}

\subsection{Feedback}\label{feedback}

Feedback is crucial for assessing the success of a project, by obtaining
feedback from testers or trial users changes can be made to improve the
experience of the application. Feedback can be obtained during
development by having people use certain aspects of the application as
it's being developed. If during feedback multiple users raise complaints
about the same aspects this is a key indicator to change that aspect of
the application before full deployment to ensure a higher level of
polish and maintain a higher level of respect and image.

\subsubsection{User Trials}\label{user-trials}

\section{Specification}\label{specification}

\subsection{Deliverables}\label{deliverables}

\begin{itemize}
\tightlist
\item
  activities must be associated with products
\item
  products may be intermediate or end deliverables
\item
  products may be technical, management or quality
\item
  products will form a hierarchy -- some MUST be accomplished before
  others can be undertaken
\end{itemize}

\subsubsection{Stages}\label{stages}

\begin{itemize}
\tightlist
\item
  A schedule of activites
\end{itemize}

\subsubsection{Risk Analysis}\label{risk-analysis}

There are many risks present with any kind of project, I will be
identifying the most relevant and predictable risks and assessing the
impact that could be caused. By identifying the risks posed I can
attempt to avoid and mitigate these risks and plan for those that I
can't control.

\begin{longtable}[]{@{}lll@{}}
\caption{List of risks.}\tabularnewline
\toprule
\begin{minipage}[b]{0.28\columnwidth}\raggedright\strut
Risks\strut
\end{minipage} & \begin{minipage}[b]{0.13\columnwidth}\raggedright\strut
Impact level\strut
\end{minipage} & \begin{minipage}[b]{0.50\columnwidth}\raggedright\strut
Reaction\strut
\end{minipage}\tabularnewline
\midrule
\endfirsthead
\toprule
\begin{minipage}[b]{0.28\columnwidth}\raggedright\strut
Risks\strut
\end{minipage} & \begin{minipage}[b]{0.13\columnwidth}\raggedright\strut
Impact level\strut
\end{minipage} & \begin{minipage}[b]{0.50\columnwidth}\raggedright\strut
Reaction\strut
\end{minipage}\tabularnewline
\midrule
\endhead
\begin{minipage}[t]{0.28\columnwidth}\raggedright\strut
Sickness\strut
\end{minipage} & \begin{minipage}[t]{0.13\columnwidth}\raggedright\strut
Low\strut
\end{minipage} & \begin{minipage}[t]{0.50\columnwidth}\raggedright\strut
Avoid getting ill. I\strut
\end{minipage}\tabularnewline
\begin{minipage}[t]{0.28\columnwidth}\raggedright\strut
Data loss\strut
\end{minipage} & \begin{minipage}[t]{0.13\columnwidth}\raggedright\strut
Low\strut
\end{minipage} & \begin{minipage}[t]{0.50\columnwidth}\raggedright\strut
Mitigate risk with multiple backups and version control.\strut
\end{minipage}\tabularnewline
\begin{minipage}[t]{0.28\columnwidth}\raggedright\strut
Project complexity\strut
\end{minipage} & \begin{minipage}[t]{0.13\columnwidth}\raggedright\strut
Medium\strut
\end{minipage} & \begin{minipage}[t]{0.50\columnwidth}\raggedright\strut
Avoid making it too complex, or too simple.\strut
\end{minipage}\tabularnewline
\begin{minipage}[t]{0.28\columnwidth}\raggedright\strut
Scope creep\strut
\end{minipage} & \begin{minipage}[t]{0.13\columnwidth}\raggedright\strut
Low\strut
\end{minipage} & \begin{minipage}[t]{0.50\columnwidth}\raggedright\strut
Avoid implementing features not outlined.\strut
\end{minipage}\tabularnewline
\begin{minipage}[t]{0.28\columnwidth}\raggedright\strut
Communication with supervisor\strut
\end{minipage} & \begin{minipage}[t]{0.13\columnwidth}\raggedright\strut
Low\strut
\end{minipage} & \begin{minipage}[t]{0.50\columnwidth}\raggedright\strut
Mitigate by keeping in regular contact.\strut
\end{minipage}\tabularnewline
\begin{minipage}[t]{0.28\columnwidth}\raggedright\strut
Learning curve\strut
\end{minipage} & \begin{minipage}[t]{0.13\columnwidth}\raggedright\strut
Medium\strut
\end{minipage} & \begin{minipage}[t]{0.50\columnwidth}\raggedright\strut
Mitigation by working with what I know\strut
\end{minipage}\tabularnewline
\bottomrule
\end{longtable}

\subparagraph{Sickness}\label{sickness}

Besides avoiding getting colds and flu which pose little risk to the
project, the only other form of impact would be broken bones etc.. This
would impact my learning however I would continue

\subparagraph{Data Loss}\label{data-loss}

I have my data backed up on two devices, a local NAS and the online
service MEGA sync \cite{mega}

\subparagraph{Project complexity}\label{project-complexity}

My project has been agreed by my supervisor and so I believe it is
neither too complex or too simple for the grade I would like to obtain.
I feel I have made my project achievable and would like to add more
functionality as time permitted.

\subparagraph{Scope Creep}\label{scope-creep}

It is very possible for scope creep to occur with my project however I
will be ensuring I complete all the features and functionality outlined
within my project proposal before attempting to expand/improve upon the
application to ensure I have a fully working project.

\subparagraph{Communication}\label{communication}

By keeping in regular contact with my superviosr I intend to be able to
get regular feedback on my performance and assistance if I need help.
Lack of communication could easily lead to a gap in what I produce and
the expectations of my supervisor and could negatively impact my final
grade.

\subparagraph{Learning Curve}\label{learning-curve}

I will be devoloping in the Java language and for a device that I am
familiar with. Although it can be tempting to work on a project in a new
language or try to implement too many features in an attempt to gain a
high mark, I feel it is more important to finalise the application and
have a fully working demonstration for submission.

\section{Research}\label{research}

your background research and the way it has influenced your project;

\subsection{Project Influence}\label{project-influence}

\subsection{Background Research}\label{background-research}

\section{Assessment of Progress}\label{assessment-of-progress}

an assessment of the progress you made, problems encountered, their
solutions and the lessons learned

\subsection{Progress Made}\label{progress-made}

\subsection{Problems Encountered}\label{problems-encountered}

\subsubsection{Solutions}\label{solutions}

\subsection{Lessons Learned}\label{lessons-learned}

\section{Accomplishments}\label{accomplishments}

The following are some of the accomplishments within developing my
application of which I am most proud.

\subsection{The Ability to Control
Lighting}\label{the-ability-to-control-lighting}

There are few applications available that allow for the connection to
and manipulation of smart-lights from within the application. Because of
my app being one of these I feel is an accomplishment.

\subsection{The Application UI Design}\label{the-application-ui-design}

Overall I feel the look and feel of my application works very well; it
adheres to the Android design guidelines \parencite{androiddesign} and
feels like a native application by maintaining a cohesive user
experience.

\subsubsection{Tab Navigation}\label{tab-navigation}

The tab navigation works well, providing the user with a simple and
clean way for using my application. By providing only the relevant
functionality for each tab the user experience is simple and intuitive.
Through avoiding too much being displayed on the screen at once,
relevant information can be seen instantly ensuring ease of use and
allowing the user to only spend as long as necessary within the
application.

Through including tab navigation it was required to use fragments within
my application, fragments are not as simple to use as a regular activity
and pose multiple design challenges over an application activity. The
extra difficulty associated with fragments did slow the initial
development of my application and my initial tab view needed to be
replaced due to the use of several deprecated styles. Despite this
challenge I feel the fragments work well within my application and the
functionality I outlined within my project has been accomplished.

\subsubsection{Snackbars and Toasts}\label{snackbars-and-toasts}

By developing for newer versions of Android I utilised the snackbar
notification system introduced in API level 23. The snackbar in most
instances replaces the original Toast notification system. Snackbars
improve upon toast in many ways as they; provide information to the user
and allow the user to interact with the notification by performing an
action if provided by the snackbar, such as undoing an action that
triggered the snackbar to be displayed. The snackbar can also be
dismissed by swiping, while the toast is displayed until the specified
display time has passed.

Toasts are stilled used within the newer versions of Android, however
they are now often used for system notification as they can be displayed
without being associated with an activity and as they can't be dismissed
are good for showing warnings or important information. Snackbars also
blend better within an application, showing up above or moving elements
such as a floating action button, ensuring that the notification can be
seen and that aspects of the view are not obstructed.

It is specified within the guidelines that a snackbar should be
displayed as long as necessary, as such short notifications should not
be displayed for a long time, while notifications that provide an action
or consist of a lot of text should be displayed for longer to allow the
user to fully read the text or provide adequate time for interaction.

\subsubsection{Using the Weather Icon
Font}\label{using-the-weather-icon-font}

The usage of the weather icon font provides multiple advantages over
that of PNG, JPG or SVG graphics. Of these graphics types JPG doesn't
allow for transparency and so would require a background colour to be
saved, this would not only increase the storage space required to store
the image, but would require the image to be edited if the background
colour of the view were to be changed, making the use of a stored
background difficult to manage. PNG does allow for transparency, however
PNG images requires a relatively larger amount of storage space when
compared with JPG.

PNG and JPG are bitmap image types, meaning they store all the pixel
data such as colour values and luminosity among other values as
essentially a three dimensional array over the colour space. Due to the
means of storage the graphics are not suitable for scaling, especially
to increase the size of the graphic as this would require the need to
generate data that does not exist; for example in an image that is
\(800 \times 800\) pixels, there would be 640,000 pixels, if the image
were scaled to \(1000 \times 1000\) pixels there would need to be
1,000,000 pixels and as a result 360,000 pixels need to be generated
from the existing image, resulting in processing artefacts, banding or
other graphical errors.

SVG is a vector graphic and consists of instructions on how to draw a
graphic and this is handled when the image is to be displayed. Due to
this flexible drawing ability, the graphic can be generated to be as
large as required with a trade off of computational time for storage,
compared to bitmap images. Android is now capable of handling both
types, however this has not always been the case. SVG support has been
officially supported for versions above 4.4 (KitKat) with many adopting
workarounds using the webview library prior to support
\parencite{svgAndroid}. As a result there are many poor implementations
of using SVG and the current advice provided in the Android guidelines
is to generate PNG graphics from SVG files as to support older devices.
When the PNG files are produced multiple sizes can be generated to
support devices of varying screen sizes and in doing so requires
multiple sizes of the same file to be saved, taking up more storage
space.

By using a TrueType Font (TTF) it is possible to gain the benefits of an
SVG without the need to generate resource files or use unsupported
implementations to display the graphics. Other benefits include the
ability to easily colour the icons by changing the font colour, the font
appears as a single file within the Android project resulting in less
clutter in the project structure. Lastly the size required to store the
font consisting of hundreds of icons is relatively small as it requires
98kb of storage while a single large PNG around 30kb.

\section{Success and Failure}\label{success-and-failure}

assessment of the success or failure of the project as a whole.

\subsection{Successes}\label{successes}

\subsubsection{Using SQLlite}\label{using-sqllite}

I had not forseen the need to use a database

\subsubsection{Pending intents}\label{pending-intents}

\subsection{Failures}\label{failures}

\subsubsection{No Adjustable alarm}\label{no-adjustable-alarm}

\subsubsection{Not Reaching Stretch}\label{not-reaching-stretch}

\section{Original Project Plan}\label{original-project-plan}

The following is the original project plan for my application.

\subsection{Aims and objectives}\label{aims-and-objectives}

What I will be developing over the upcoming months is an app to help
users get out of bed easier in the morning in a useful and information
rich way.

My aims are:

\begin{itemize}
\tightlist
\item
  Produce an alarm app with all the functionality users are used to.
\item
  Integrate Smart bulb functionality into the app to turn the light on
  in the morning with the alarm.
\item
  To turn off the lights at night without having to get out of bed.
\item
  Provide weather information for the day.
\item
  Inform the user of their schedule for the day and upcoming events.
\item
  Publish the application to the play store for download and use by
  others.
\end{itemize}

\subsection{Stakeholders}\label{stakeholders}

I have identified the following stakeholders:

\begin{itemize}
\item
  Myself - Not only am I developing the application making me a
  stakeholder, I am also very interested in home automation and waking
  up happy.
\item
  My supervisor, Marcus Winter - By accepting to be my supervisor Marcus
  is also a stakeholder for my application.
\item
  The second reader - Will also be involved and will be grading my
  project.
\item
  An expanding user base of smart bulbs - Although the market currently
  is small the cost of smart bulbs is decreasing making them more
  available to users.
\item
  Anyone that uses an alarm - The largest stakeholder I have is anyone
  that uses an alarm on the Android platform. More specifically those
  who own smart bulbs or other connected devices.
\end{itemize}

\subsection{Communications}\label{communications}

I will maintain contact with my project supervisor with monthly meetings
where I intend to measure my progress against deadlines and goals,
reflect on what progress has been made and address issues, challenges
and for development advice to assist me successfully complete my project
as planned.

Regular emails will also be used between meetings to keep in contact and
keep my supervisor informed of what I intend to talk about and on my
progress made.

\subsection{Installation Process}\label{installation-process}

I would like to publish my application to the Play store once developed
to a satisfactory level. To install an app from the Play store you can
simply select the option to install the app and it will be downloaded
and installed seamlessly.

I will be ensuring to develop to a high standard and ensure there are no
issues, bugs or flaws with my application.

\subsection{Quality checks}\label{quality-checks}

During development I will ensure to maintain my code and follow the
principles that have been taught to me and that I have learned and will
learn, in doing so my code should be easily maintainable, readable and
extendable for possible extensions and stretch goals.

I will develop a test plan as I continue to develop my application to
allow me to note issues and ensure previous functionality has not been
effected by further developments.

\subsection{How will I measure
success?}\label{how-will-i-measure-success}

My key performance indicators are outlined below:

\begin{itemize}
\tightlist
\item
  Alarm functionality
\item
  Smart bulb integration
\item
  Weather functionality
\item
  Calendar Integration (Stretch)
\item
  Text to speech (Stretch)
\end{itemize}

If I am unable to produce a working alarm app with smart bulb
functionality I will have failed to achieve what I intended to develop
and so these are my highest priority.

\subsection{Challenges}\label{challenges}

There are many challenges I will face during my project these include
the following:

\begin{itemize}
\item
  Making my application extensible to other home automation systems such
  as the \cite{belkinwemo}. This would allow for greater flexibility and
  a larger target audience.
\item
  Handling various devices, there are now many various appliances that
  are connected to home automation, from washing machines and fridges to
  door locks and CCTV. Many devices could be useful for morning and
  night automation, such as closing the curtains at night or turning the
  kettle on in the morning. Each of these devices will have different
  functions and are very different from one another. To be able to add
  several and include them within a scene/scenario to perform multiple
  actions would be a very powerful inclusion within the app.
\item
  Providing a good level of home automation without making the app
  cumbersome or difficult to use. My target audience is average users
  who would like to wake up easier and more refreshed and have their
  days information available straight away. They don't want to spend
  ages setting up the light bulbs or struggling to find settings as they
  will get fed up and stop using and potentially uninstall the app.
\end{itemize}

\section{Divergence From Original
Plan}\label{divergence-from-original-plan}

No project goes through all the stages of development and remains
exactly as outlined during the first stages of development. This is
attributed to many factors that may not have been apparent at the onset
of development, many changes are small involving adjustments to the UI
to improve user experience; these changes can be seen in completed
products also, with many applications receiving design overhauls such as
the introduction of the ribbon design to the Microsoft office suite for
office 2007 \parencite{officeribbon} and the regularly changing design
philosophy of Android.

\subsection{User Testing/Evaluation}\label{user-testingevaluation}

Through getting others to use my application I was able to gain useful
feedback to alter the design and functionality of my application to
improve upon the designs.

Due to the need for a hue lighting bridge to be available, it was not
always possible to obtain hands on feedback from users and instead
required the need to obtain feedback through the use of screenshots and
describing the actions available to the user; through this limitation it
was not possible to gain the most feedback possible, but should provide
a good insight into the design and enhancements for the application.

\subsubsection{Inclusion of the Light
Icon}\label{inclusion-of-the-light-icon}

Several users suggested the inclusion of a lighting icon to indicate the
type of light being listed, this is because of the ability for lights to
be renamed and could lead to confusion in the future.

The icon provides at a glance a hint to the type of light being shown
and can assist the user. Within the application the icons used are
freely available and provided by Philips under the development program
to promote development and use of the Hue lighting system
\parencite{hueicons}.

\subsubsection{Greater Control of
Lights}\label{greater-control-of-lights}

Many users suggested a greater level of lighting control within the
application, features such as; the ability to connect with multiple
bridges, add new lights to the system, adjust brightness and colour.
Each of these features could make for a more enjoyable user experience
and allow the users to perform many function from within the
application, however due to complexity in certain aspects such as
allowing for multiple bridges, these were not possible to be included
within the application.

Features such as the brightness and hue controls were not included as a
design limitation, with the belief that adding these features would add
an extra level of complexity of use that was not intended for the
application. During the conceptual phase of development a goal for a
clean and minimal design was set, resulting in simplicity of use.

I would like to include these controls if possible in the future but I
would like to find a user friendly means of providing the controls. Hue
adjustment would be more complex than the inclusion of a brightness
adjustment as brightness can be linearly controlled while the hue
involves; hue, saturation, colour temperature and the use of the CIE
colour space \parencite{1475-4878-30-4-301}. Due to this complexity it
is unlikely to be included within my application in the near future.

\section{Further Areas of Investigation and
Enhancements}\label{further-areas-of-investigation-and-enhancements}

\subsection{Further Investigation}\label{further-investigation}

\subsection{Enhancements}\label{enhancements}

Other smart-devices not just lights.

\section{Evaluation}\label{evaluation}

A critical evaluation of every significant area of your project work,
including your choice of project and how it fits in with the modules you
have studied;

\subsection{Choice of Project}\label{choice-of-project}

\subsection{Course Relation}\label{course-relation}


\pagebreak
\printbibliography

All links were last followed on the 9\textsuperscript{th} of May, 2017
\pagebreak

\appendix

\section{\\Application Designs}\label{apdx:designs}

\subsection{Hub Set-up}

\begin{figure}[H]
  \centering
  \subfloat[The main hub set-up menu]{{\includegraphics[scale=0.2]{Images/designs/AppDesign_01.png}}}
  \qquad
  \subfloat[List of available hubs]{{\includegraphics[scale=0.2]{Images/designs/AppDesign_02.png}}}

  \label{fig:hubdesignview}
  \caption{The views for configuring the hub.}
\end{figure}

\begin{figure}
 \centering
  \includegraphics[scale=0.2]{Images/designs/AppDesign_03.png}
  \caption{Push-link prompt}
\end{figure}


\subsection{Light Tab}

\begin{figure}[H]
\centering
\includegraphics[scale=0.2]{Images/designs/AppDesign_04.png}
\caption{The design of the light tab}
\end{figure}

\subsection{Alarm Tab}

\begin{figure}[H]
\centering
\includegraphics[scale=0.2]{Images/designs/AppDesign_05.png}
\caption{The design of the alarm tab}
\end{figure}

\subsection{Weather Tab}

\begin{figure}[H]
  \centering
  \subfloat[The top view of the weather tab]{{\includegraphics[scale=0.2]{Images/designs/AppDesign_06.png}}}
  \qquad
  \subfloat[A scrolled down view of the weather tab]{{\includegraphics[scale=0.2]{Images/designs/AppDesign_07.png}}}


  \label{fig:weatherdesign}
  \caption{A view of the weather tabs.}
\end{figure}


\section{\\Application Documentation}
The following documentation provides instructions on how to install and
use the application.

\subsection{Installation}\label{installation}

To install an Android application not available from the play store,
there are several steps that are required to install the app.

\subsubsection{Enable Unknown Sources}\label{enable-unknown-sources}

To install the application to a physical device it is necessary to
enable the setting to allow for installation of apps from unknown
sources; this is required as the application is not available from the
play store. The setting to allow unknown sources can be found in:
Settings \(\rightarrow\) Security \(\rightarrow\) Unknown sources:

\begin{figure}[H]
  \centering
  \subfloat[Security and Fingerprint]{{\includegraphics[scale=0.1]{./Images/security.png}}}
  \qquad
  \subfloat[Unknown sources enabled]{{\includegraphics[scale=0.1]{./Images/unknownsources.png}}}

  \caption{Enabling unknown sources}%

  \label{fig:unknownsources}
\end{figure}

\subsubsection{Copy APK to Device and install
locally}\label{copy-apk-to-device-and-install-locally}

Once installation from unknown sources has been enabled, copy the
application APK to the device. On the device navigate to the location
that APK was copied to within the file browser on the device. Tap on the
APK and an installation prompt will be presented; accept the
installation and wait for the app to be installed.

When installation has been completed the app will now be available from
within the application draw or on the home-screen.

\subsubsection{Alternative Installation}\label{alternative-installation}

It is possible to install the application using the command line; in
this example it shows doing so using Windows 10 and the command line.

Before being able to install the application it is not only necessary to
enable unknown sources as above, but also to enable ADB debugging.

\subsubsection{Enable ADB on Device (Alternative
Installation)}\label{enable-adb-on-device-alternative-installation}

Enabling ADB can be done by first enabling developer options, go to
Settings \(\rightarrow\) About phone. From here you will need to tap the
\emph{Build number} around 7 times in succession, after which a success
message will show you have enabled developer settings.

\begin{figure}[H]
  \centering
  \subfloat[Enabling developer options]{{\includegraphics[scale=0.6]{./Images/enablingDeveloper.png}}}
  \qquad
  \subfloat[Developer options enabled message \label{fig:developerAvailable}]{{\includegraphics[scale=0.6]{./Images/developerSuccess.png}}}
  \caption{Enabling developer options}
  \label{fig:enablingDeveloperoptions}
\end{figure}

Developer settings are now available from within the settings menu as
shown in figure \ref{fig:developerOption}. Enter this option and scroll
to the option \emph{Android debugging} as shown in figure
\ref{fig:adbOption}, when toggled a warning message will appear (figure
\ref{fig:enableAdbMsg}); read and accept the warning to enable ADB.

\begin{figure}[H]
  \centering
  \subfloat[Developer option is now available \label{fig:developerOption}]{{\includegraphics[scale=0.1]{Images/developerOption.png}}}
  \qquad
  \subfloat[Allow ADB message \label{fig:enableAdbMsg}]{{\includegraphics[scale=0.5]{./Images/allowAdbMsg.png}}}
  \caption{Enabling ADB on device}
  \label{fig:enablingADB}
\end{figure}

\begin{figure}
  \centering
  \includegraphics[scale=0.1]{Images/adbDebuggingOption.png}
  \caption{ADB option in developer menu}
  \label{fig:adbOption}
\end{figure}

\subsubsection{\texorpdfstring{Install ADB - \emph{Windows 10}
(Alternative
Installation)}{Install ADB - Windows 10 (Alternative Installation)}}\label{install-adb---windows-10-alternative-installation}

The installation of ADB will often be included with Android studio,
however to install ADB alone the easiest method would be to download and
use the ADB installer tool found on
\href{https://forum.xda-developers.com/showthread.php?t=2588979}{XDA-Developers}
\cite{AdbInstaller}.

Once installed running \lstinline!adb devices! will show all devices
connected to the computer with ADB enabled as shown in figure
\ref{fig:adbDevices}.

\begin{figure}
  \centering
  \includegraphics[scale=0.5]{Images/adbDevices.png}
  \caption{Results of running ADB devices command}
  \label{fig:adbDevices}
\end{figure}

\subsubsection{Installation via ADB (Alternative
Installation)}\label{installation-via-adb-alternative-installation}

Now that ADB is enabled on the device and installed on the computer it
is possible to run the command
\lstinline!adb install *apk file location*! passing in the location to
the apk file. If successful an installtion success message will be
displayed as shown in figure \ref{fig:adbSuccess}.

\begin{figure}[H]
  \centering
  \subfloat[ADB install command]{{\includegraphics[scale=0.5]{./Images/adbInstall.png}}}
  \qquad
  \subfloat[Results of application installation success]{{\includegraphics[scale=0.5]{./Images/adbInstallSuccess.png}}}
  \caption{Installing APK using ADB}
  \label{fig:adbSuccess}
\end{figure}

\subsection{Alarms}\label{alarms}

This section will provide instructions for the functionality of the
alarm functionality.

\subsubsection{Create}\label{create}

By clicking on the alarm create button at the bottom of the screen (as
seen in figure \ref{fig:addAlarm}) a time picker dialog will appear.

\begin{figure}[H]
  \centering
  \includegraphics[scale=0.1]{Images/addAlarm.png}
  \caption{Create alarm button}
  \label{fig:addAlarm}
\end{figure}

First the hour will be selected, to select the hour desired for the
alarm either tap on time or for more precision, press and drag the
\emph{hand} of the clock to the desired time. When released the picker
will now allow for the selection of the minutes; repeat the same action
as for the hour to select the chosen time.

\subsubsection{Rename}\label{rename}

Renaming an alarm is simple, by taping the alarm that needs renaming an
edit text pop-up will appear (as seen in figure \ref{fig:changeLabel});
enter the new label desired and confirm the change, the alarm is now
called something else.

\begin{figure}[H]
  \centering
  \includegraphics[scale=0.1]{Images/changeLabel.png}
  \caption{Change label edit text}
  \label{fig:changeLabel}
\end{figure}

The label provided to the alarm will appear when the alarm goes off and
be displayed in the alarm notification.

\subsubsection{Turning Off/On an Alarm}\label{turning-offon-an-alarm}

By pressing the switch on the right hand side of the screen, it is
possible to toggle the alarm on or off.

\begin{figure}[H]
  \centering
  \includegraphics[trim= 0 2000 0 0, clip, scale=0.1]{Images/addAlarm.png}
  \caption{Toggle alarm switch}
  \label{fig:toggleAlarm}
\end{figure}

\subsubsection{Delete}\label{delete}

To delete and existing alarm, simply press and hold the alarm to be
deleted, a prompt will appear asking to confirm the action.

\begin{figure}[H]
  \centering
  \includegraphics[scale=0.2]{Images/deleteAlarmDialog.png}
  \caption{Delete alarm dialog box}
  \label{fig:deleteAlarm}
\end{figure}

\subsection{Lights}\label{lights}

It is required to connect the application to a lighting bridge to be
able to enable the lighting functionality within the application. The
alarms and weather will work without the lighting being configured.

\subsubsection{Connecting to the Bridge}\label{connecting-to-the-bridge}

To be able to use the smart-light functionality it is necessary to pair
the application with the bridge prior to use, this allows the
application to be \emph{white-listed} and provide access to the lighting
interface.

When navigating to the lighting tab, if no bridge has been configured a
text prompt will appear and at the press of the button will take you to
the set-up activity. If the activity does not automatically begin
searching for a bridge, pressing the search for bridge button will begin
a search.

\begin{figure}[H]
  \centering
  \includegraphics[scale=0.1]{Images/setupPrompt.png}
  \caption{Bridge set-up prompt screen}
  \label{fig:setupPrompt}
\end{figure}

If no bridges are returned please ensure you are on the same local
network as the bridge and there are no firewalls, VPNs or other
potential network configuration that could be blocking communications.
If no bridges are found or you would like to search again, simply press
the search button again.

\begin{figure}[H]
  \centering
  \subfloat[Searching for bridge]{{\includegraphics[scale=0.09]{./Images/searchingForHub.png}}}
  \qquad
  \subfloat[Bridge connection error]{{\includegraphics[scale=0.2]{Images/bridgeError.png}}}
  \caption{Finding available bridges}%
  \label{fig:searchForBridge}%
\end{figure}

When the search is completed a list of all found bridges will be
displayed, to select one tap on it to begin the authentication process.

\begin{figure}[H]
  \centering
  \subfloat[No bridge found screen]{{\includegraphics[trim= 0 2000 0 0, clip, scale=0.09]{./Images/noBridgeFound.png}}}
  \qquad
  \subfloat[List of available bridges]{{\includegraphics[trim= 0 2000 0 0, clip, scale=0.09]{./Images/bridgeList.png}}}
  \caption{Finding available bridges}
  \label{fig:findBridge}%
\end{figure}

On selecting a bridge the push-button authentication will occur, please
press the large push-link button on the front/top of the bridge. This
action is to ensure that there is physical access to the bridge as a
security measure to prevent un-authorised applications or intruders on
the network gaining access to the lighting interface.

\begin{figure}[H]
  \centering
  \includegraphics[scale=0.09]{Images/pushLinkScreen.png}
  \caption{The push-link prompt and countdown}
  \label{fig:pushPrompt}
\end{figure}

On successful authentication you shall be returned back to the main
application, now when navigating to the lighting tab any connected
lights shall be displayed similar to that shown in figure
\ref{fig:lightList}.

\begin{figure}[H]
  \centering
  \includegraphics[scale=0.1]{Images/lightList.png}
  \caption{List of available lights}
  \label{fig:lightList}
\end{figure}

\subsubsection{Toggling the lights}\label{toggling-the-lights}

Any lights assigned to the bridge will be displayed in a list,
displaying their name and an icon indicating the light type.

Each alarm can be turned on/off easily by tapping the switch on the
right hand side of the screen.

It should be noted that if a light is turned off at the wall/switch the
last known light state will be displayed within the application when
displayed.

\subsection{Weather}\label{weather}

Weather will be displayed on the weather tab. To change the location of
the weather being displayed, press on the menu expansion button from
anywhere within the application as seen in figure \ref{fig:setcity},
this will display the option to change the city stored.

\begin{figure}[H]
  \centering
  \subfloat[The options menu is in the top right]{{\includegraphics[scale=0.1]{./Images/optionmenu.png}}}
  \qquad
  \subfloat[Change city option]{{\includegraphics[scale=0.1]{./Images/changecityoption.png}}}
  \caption{The city selection option}%
  \label{fig:setcity}%
\end{figure}

when the change city option is selected a text dialog will appear
allowing for the entry of a different city, simply type in the name of
the city desired and accept the change, the weather will be updated when
the weather tab is loaded next.



\section{\\Activity Log}
\input{log}

%%%%%%%%%%%%%%%%%%%%%%%%%
\end{document}
