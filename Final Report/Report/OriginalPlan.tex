\section{Original Project Plan}\label{original-project-plan}

The following is the original project plan for my application.

\subsection{Aims and objectives}\label{aims-and-objectives}

What I will be developing over the upcoming months is an app to help
users get out of bed easier in the morning in a useful and information
rich way.

My aims are:

\begin{itemize}
\tightlist
\item
  Produce an alarm app with all the functionality users are used to.
\item
  Integrate Smart bulb functionality into the app to turn the light on
  in the morning with the alarm.
\item
  To turn off the lights at night without having to get out of bed.
\item
  Provide weather information for the day.
\item
  Inform the user of their schedule for the day and upcoming events.
\item
  Publish the application to the play store for download and use by
  others.
\end{itemize}

\subsection{Stakeholders}\label{stakeholders}

I have identified the following stakeholders:

\begin{itemize}
\item
  Myself - Not only am I developing the application making me a
  stakeholder, I am also very interested in home automation and waking
  up happy.
\item
  My supervisor, Marcus Winter - By accepting to be my supervisor Marcus
  is also a stakeholder for my application.
\item
  The second reader - Will also be involved and will be grading my
  project.
\item
  An expanding user base of smart bulbs - Although the market currently
  is small the cost of smart bulbs is decreasing making them more
  available to users.
\item
  Anyone that uses an alarm - The largest stakeholder I have is anyone
  that uses an alarm on the Android platform. More specifically those
  who own smart bulbs or other connected devices.
\end{itemize}

\subsection{Communications}\label{communications}

I will maintain contact with my project supervisor with monthly meetings
where I intend to measure my progress against deadlines and goals,
reflect on what progress has been made and address issues, challenges
and for development advice to assist me successfully complete my project
as planned.

Regular emails will also be used between meetings to keep in contact and
keep my supervisor informed of what I intend to talk about and on my
progress made.

\subsection{Installation Process}\label{installation-process}

I would like to publish my application to the Play store once developed
to a satisfactory level. To install an app from the Play store you can
simply select the option to install the app and it will be downloaded
and installed seamlessly.

I will be ensuring to develop to a high standard and ensure there are no
issues, bugs or flaws with my application.

\subsection{Quality checks}\label{quality-checks}

During development I will ensure to maintain my code and follow the
principles that have been taught to me and that I have learned and will
learn, in doing so my code should be easily maintainable, readable and
extendable for possible extensions and stretch goals.

I will develop a test plan as I continue to develop my application to
allow me to note issues and ensure previous functionality has not been
effected by further developments.

\subsection{How will I measure
success?}\label{how-will-i-measure-success}

My key performance indicators are outlined below:

\begin{itemize}
\tightlist
\item
  Alarm functionality
\item
  Smart bulb integration
\item
  Weather functionality
\item
  Calendar Integration (Stretch)
\item
  Text to speech (Stretch)
\end{itemize}

If I am unable to produce a working alarm app with smart bulb
functionality I will have failed to achieve what I intended to develop
and so these are my highest priority.

\subsection{Challenges}\label{challenges}

There are many challenges I will face during my project these include
the following:

\begin{itemize}
\item
  Making my application extensible to other home automation systems such
  as the \cite{belkinwemo}. This would allow for greater flexibility and
  a larger target audience.
\item
  Handling various devices, there are now many various appliances that
  are connected to home automation, from washing machines and fridges to
  door locks and CCTV. Many devices could be useful for morning and
  night automation, such as closing the curtains at night or turning the
  kettle on in the morning. Each of these devices will have different
  functions and are very different from one another. To be able to add
  several and include them within a scene/scenario to perform multiple
  actions would be a very powerful inclusion within the app.
\item
  Providing a good level of home automation without making the app
  cumbersome or difficult to use. My target audience is average users
  who would like to wake up easier and more refreshed and have their
  days information available straight away. They don't want to spend
  ages setting up the light bulbs or struggling to find settings as they
  will get fed up and stop using and potentially uninstall the app.
\end{itemize}

\section{Divergence From Original
Plan}\label{divergence-from-original-plan}

together with any later versions or a discussion of any necessary
changes to the plan. We recommend a count of 5000 words. This report is
an accessible component of the project and is one the examiners will pay
close attention to. Please hand in one copy by 18th May 2017. All
reports MUST contain a first page with student name, student number,
exit award for which you are registered and a short title.
