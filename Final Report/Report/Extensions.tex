\section{Further Areas of Investigation and
Enhancements}\label{further-areas-of-investigation-and-enhancements}

During this project there are areas that given time and resources would
like to have investigated or developed further.

\subsection{Further Investigation}\label{further-investigation}

The following are areas that provided more time, would have been
investigated into further.

\subsubsection{Android Platform}\label{android-platform}

While developing my application for the Android platform it became clear
there are plenty of libraries available for use. This can be attributed
to not only being based upon Java, with Java having a large code base
and many libraries included and even more available by importing
external libraries such as; Retrofit, that provides a type-safe REST
client for Android providing mappings into a client interface using
annotations, dialogplus provides the ability to display dialog items in
a pleasant Lollipop styles animation, with many more available
\parencite{androidlibs}.

As the space of development is so large I would often conceive of a plan
to resolve issues or implement functions within my application that may
not have been the optimum solution. Through further experience with the
platform and further time for development several solutions I
implemented could have been implemented more efficiently, safer and more
stable through the use of Android libraries or external libraries known
to be improve development speed and code readability by addressing
common Android development paradigms and styles.

\subsubsection{iOS and Other Platforms}\label{ios-and-other-platforms}

Though Android makes the majority of the current mobile platform, iOS is
second taking around a 12.5\% market share \parencite{osMarketSahre} and
so should be considered when developing applications. It has been found
that iOS users often will spend more on and within applications making
the platform potentially more lucrative over others according to details
from 2013 \parencite{appMoney}.

iOS applications can be written in several languages such as; Swift,
C++, C Objective-C and more \parencite{iosDeveloperLanguage}. Much like
the Android studio integrated development environment (IDE), iOS
applications can be written within Xcode; Apple's own IDE. The benefits
of using the IDEs is a tool-chain useful for developing, debugging and
designing applications.

It would clearly be beneficial to be able to develop an application once
and be able to deploy it across multiple platforms and systems. By
leveraging systems such as \cite{cordova} it is possible to develop an
application once and have it wrapped in a native container, this is then
able to be distributed on respective platforms application store.

Cordova is able to use many native APIs available on the platforms being
targeted and in doing so is able to blend with the native platform and
appear like a regular app. There are APIs that are unavailable through
Cordova however and so there may be a need to target a platform
directly. Other benefits of building natively is the ability to write
code at a lower level, for instance Android applications can be written
in C and C++, allowing for lower level development which can be
beneficial if applications are being written for lower powered devices
or further performance is to be desired from the application.

\subsubsection{Functional Programming for Mobile
Platforms}\label{functional-programming-for-mobile-platforms}

Being able to leverage features available within functional programming
would allow for the ability to produce safe, concise and more reasonable
code. The benefits of which are numerous, allowing for code to become
more modular and increasing re-use and prevent potential issues when
running multi-threaded functions through the inclusion of thread-safety.

Java 8 includes several features that are provided to developers to
enable the use of such functional and so being able to include these
within my application would be beneficial. Unfortunately Java 8 is
limited to the latest version of Android and so would reduce the target
market for my application. In the future however as the market moves
towards the newer versions it would be possible to re-work aspects of my
application to a more functional style.

\subsection{Enhancements}\label{enhancements}

Due to the limited timeshare to develop my application and the other
projects and work undertaken throughout the year there are many aspects
of my application that given the time, I would like to have implemented
or improved.

\subsubsection{Integration With Other
Smart-devices}\label{integration-with-other-smart-devices}

One key area of enhancement would be to allow for my application to
interact with other smart-devices that may be available around the home,
anything from heating and kettles to a smart-toaster. By integrating
more devices into my application this could allow for a greater level of
home automation and make the morning work around the user and reduce the
amount of interaction the user would be required to make in the morning.

\subsubsection{Improve the Weather
functionality}\label{improve-the-weather-functionality}

The weather functionality within my app is limited to presenting the
users with the weather of their specified city. Preferably the ability
to obtain the users current location to pull the relevant weather would
have worked better, though this would have required requesting the user
to allow permissions for location tracking, something the user may not
want to provide to an alarm application.

Other aspects of the weather that could use improvement would be to
allow users to pick the source of their weather data as there multiple
sources of weather data and each provide varying weather accuracy. This
was not implemented due to API restrictions presented by many of the
sources, limiting API usage for free accounts or requiring the need to
pay for use. As this application is not currently intended for public
use I opted for the OpenWeatherMaps (OWM) API
\parencite{openweathermap}.

The weather icon font used within my application provides a neat way of
displaying scalable graphics that can easily be displayed in different
colours and reduce the number of graphics imported within my
application, reducing the size of the application and clutter within the
project structure during development. The icons fit well within the
material design outlined by the Android design guidelines making them
ideal for use within my application.

As each weather icon is presented using different character codes and
these codes are stored within the strings.xml file within my application
and the reference for each string includes the weather code returned by
OWM, this is used to display the appropriate weather icon. Each weather
API may return different weather codes however and due to this it is not
possible to request the icon be displayed with a weather code provided
by a different API. This could be implemented by providing an interface
that can obtain the weather from various weather APIs and have a
standardised weather code returned, though this would have taken quite
some time as OWM returns over 90 different weather codes with more
possible through the use of modifiers, for example; sunny with cloud or
cloudy at night. As there are so many different weather codes to be
handled I felt the time would be better spent on other aspects of my
application.

\subsubsection{Increase Focus of An Intelligent
Morning}\label{increase-focus-of-an-intelligent-morning}

Initially the concept for the application came from a desire for a
greater level of home automation and a more useful alarm.

Currently there is a strong focus for contextual and relevant
information being presented to the user as and when they need it. An
example of this is if a user had a flight booked, through the use of
Google Now/Assistant Google maps could recommend the time you would need
to leave to get to the airport and the best route to take, once the user
is at the airport their boarding pass can be easily accessed and any
delays can be pushed to the user via notifications so the information is
presented to the user and doesn't require any effort on behalf of the
user.

This focus towards improving the user experience by reducing the effort
required by the user is of interest and I feel there is a lot of work
that can be done to improve such functionality. Currently these
artificial assistants require some user interaction to see the
information being presented; integrating my alarm application with the
user schedule however it would be possible to present useful information
after the alarm has been turned off. This information could be presented
using text to speech (TTS) or by the use of a bold and brief
notification being displayed on the screen.

\subsubsection{Implement more smart-bulb
functionality}\label{implement-more-smart-bulb-functionality}

Currently my application allows for lights to be turned off and on
however I would like to increase the functionality to match that
provided by the manufacturer, this could include grouping lights,
changing their names for easier identification but most essential would
be the ability to adjust the brightness and hue of the lighting. I was
unable to implement this before the deadline as there were other key
aspects of my application that required work and refinement. Currently
my application supports only Philips hue lighting systems, however the
current product range for the Hue system alone covers 33 products of
varying functionality and handling each in a user friendly way posed a
challenge on presenting only the relevant options. By providing a simple
switch to turn the light on or off however accounted for all of the
products and in a way that fits within the Android platform well.

\subsection{Open-Source}\label{open-source}

In the future it would be beneficial to make my application open-source,
by allowing others access to the source code of my application and allow
for others to pull and edit my application; either to create a new fork
and work on a project based upon my application or to continue
developing Chrono-lux directly and improve upon the work.

The alarm application itself can be improved upon, however the key
interest would be in adoption of more connected devices, allowing others
to adapt the application to include their own devices not yet supported
and allowing improvements to the application for everyone is possibly
the most desired outcome of open-source software.
