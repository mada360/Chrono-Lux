\section{Evaluation}\label{evaluation}

The following is a critical evaluation of every significant area my
project, including my choice of project and how it fits in with the
modules that I have studied over the course of my degree.

\subsection{Choice of Project}\label{choice-of-project}

Deciding upon a project that was challenging and feasible for the amount
of time allocated for completion was quite a difficult task. Many ideas
and concepts that I thought of were either already well investigated or
posed the risk of being incomplete by the final deadline. I proposed
several ideas for a final year project to Marcus Winter, my project
supervisor and he helped me decide upon my home automation alarm as the
project that I should undertake.

The project I decided upon however addressed less of a computer science
research project and instead is more focused on user design and
interaction.

My reasons for the choice of my project are due to my interest in mobile
applications development and in the concept of home automation. I'm
particularly interested in a future in which many common actions are
handled transparently by automated systems, such as opening and closing
curtains and adjusting lighting as required by the user without the need
for direct user interaction and many other situations that can provide
benefits by saving time each day, to assisting those who are unable to
live alone and require regular assistance and house calls.

Overall I feel the project I chose has been challenging and is suited
towards my degree and the modules that I have undertaken.

\subsection{Course Relation}\label{course-relation}

Over the course of my degree I have undertaken multiple modules many of
which are relevant to my final project.

\subsubsection{Mobile Application
Development}\label{mobile-application-development}

Mobile app development is one module that clearly relates to my final
year project having worked on a mobile application for my project.
Throughout the module issues such as platform independent issues and
general principles for mobile user interaction design and the
constraints of developing for a mobile platform.

The issues surrounding mobile web vs native applications were also
addressed, deciding on leveraging native tools by developing directly
for a platform or for the development of a cross platform application.

There are also many platform specific issues from the development
environment and tool-chain, to the architecture and APIs provided that
need to be addressed throughout the development and maintaining of the
application.

I have worked with both physical and virtual devices and have been able
to see the benefits and restrictions of both, such as leveraging the
speed of a physical device both in installation and application usage,
over the ability to emulate numerous devices and test the application in
ways that could be very costly and time consuming.

Many requirements of most applications today are the need for;
persistence and storage, multimedia, interacting with hardware sensors
and web services. Arguably most important of all the deployment to the
relevant platforms application store was also addressed, allowing an
application to go from a development build to a signed and globally
installable application.

Each of the mentioned aspects of the module influenced my design and
development choices.

\subsubsection{Programming languages, concurrency and client server
computing}\label{programming-languages-concurrency-and-client-server-computing}

As Android is based upon the Java development language, a language that
has been taught and worked with throughout the course, much of what has
been learned has been put into practice within my project. Many
programming principles from the SOLID principles influencing my
development choices to ensure the code being written wasn't fragile and
could be easily extended in the future.

UI thread and concurrency are both key when developing for Android as
there is a focus around ensuring only UI based activities take place
within the UI thread, this is to prevent lock-ups and ensure that the
user experience remains fluid and feels fast. Through the use of
asynchronous tasks and other features available within Java, it is
possible to run activities over multiple threads or cores, something
that is becoming increasingly more important as the speed of cores
remains the same or slower each year, with an increased focus on lower
energy usage cores that can be activated as needed.

Java 8 provides many useful features to take advantage of aspects of
functional programming, this can allow for easier implementation of
concurrency within the application with the use of lambda expressions.
Unfortunately the use of Java 8 is only available in the latest version
of Android and as such, reduces the target market when developing. It is
possible to write for both Java 8 and the same code for older versions
of Android, however this would add unneeded complexity to my
application.

RESTful APIs were covered within the module and proved useful for the
development of my application with many APIs available today using
RESTful interfaces including the Philips hue and weather API.

\subsubsection{Project management}\label{project-management}

Throughout the development of my application project management has been
a useful skill to have, allowing for a balance between multiple projects
and the insight to set goals and milestones to be able to evaluate the
project during development.

By addressing issues outlined through missed deadlines, it is possible
to reduce the impact of certain aspects of development, such as scope
creep, or too much time being spent on one aspect while others go
without being addressed.

\subsubsection{User design and product
evaluation}\label{user-design-and-product-evaluation}

My project focused mainly on the user design and so this module was
helpful in how to perform user evaluation and to address not only the
aspects that could do with improvement but also those that worked well,
providing a useful feedback loop during development, allowing me to work
towards the strengths of the application design and re-work aspects that
could be improved.
