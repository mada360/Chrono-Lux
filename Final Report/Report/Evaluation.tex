\section{Evaluation}\label{evaluation}

The following is a critical evaluation of every significant area my
project, including my choice of project and how it fits in with the
modules that I have studied over the course of my degree.

\subsection{Choice of Project}\label{choice-of-project}

Deciding upon a project that was challenging and feasible for the amount
of time allocated for completion was quite a difficult task. Many ideas
and concepts that I thought of were either already well investigated or
posed the risk of being incomplete by the final deadline. I proposed
several ideas for a final year project to Marcus Winter, my project
supervisor and he helped me decide upon my home automation alarm as the
project that I should undertake.

The project I decided upon however addressed less of a computer science
research project and instead is more focused on user design and
interaction.

My reasons for the choice of my project are due to my interest in mobile
applications development and in the concept of home automation. I'm
particularly interested in a future in which many common actions are
handled transparently by automated systems, such as opening and closing
curtains and adjusting lighting as required by the user without the need
for direct user interaction and many other situations that can provide
benefits by saving time each day, to assisting those who are unable to
live alone and require regular assistance and house calls.

Overall I feel the project I chose has been challenging and is suited
towards my degree

\subsection{Course Relation}\label{course-relation}

Over the course of my degree I have undertaken multiple modules many of
which are relevant to my final project.

\subsubsection{Mobile Application
Development}\label{mobile-application-development}

Mobile app development is one module that clearly relates to my final
year project having worked on a mobile application for my project.
Throughout the module issues such as platform independent issues and
general principles for mobile user interaction design and the
constraints of developing for a mobile platform.

The issues surrounding mobile web vs native applications were also
addressed, deciding on leveraging native tools by developing directly
for a platform or for the development of a cross platform application.

There are also many platform specific issues from the development
environment and tool-chain, to the architecture and APIs provided that
need to be addressed throughout the development and maintaining of the
application.

I have worked with both physical and virtual devices and have been able
to see the benefits and restrictions of both, such as leveraging the
speed of a physical device both in installation and application usage,
over the ability to emulate numerous devices and test the application in
ways that could be very costly and time consuming.

Many requirements of most applications today are the need for;
persistence and storage, multimedia, interacting with hardware sensors
and web services. Arguably most important of all the deployment to the
relevant platforms application store was also addressed, allowing an
application to go from a development build to a signed and globally
installable application.

\subsubsection{Programming languages, concurrency and client server
computing}\label{programming-languages-concurrency-and-client-server-computing}

As Android is based upon the Java development language

Java

UI thread and concurrency

RESTful

\subsubsection{Project management}\label{project-management}

\subsubsection{User design and product
evaluation}\label{user-design-and-product-evaluation}
