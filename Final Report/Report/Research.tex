\section{Research}\label{research}

Aspects of my project were influenced by the research collected before
and during development, resulting in changes from what was originally
planned.

\subsection{Philips Hue API}\label{philips-hue-api}

My project utilises the Philips Hue connected light bulbs, the
\cite{philipshue} system uses the \cite{zigbee} standard, though all of
this is transparent as the method of interfacing with the devices is to
use `GET', `PUSH', `POST' and `PUT' URL requests and provide JSON
formatted commands in the body to interact.

The state of a specific light can be received using `GET' and providing
the URL /api/devID/lights/1 or all of the lights by not specifying the
number.

The state can be changed using `PUT' instead and providing attributes
and their values that you would like to change, for example:

\begin{lstlisting}
{"on":true, "bri":255}
\end{lstlisting}

A few useful attributes for my application are: on = true/false bri =
Brightness between 0 and 254

Colour settings include: sat = Saturation between 0 and 254 hue = The
hue of the light (hue runs from 0 to 65535)

Through researching into the Hue API further there existed a Hue
Software Development Kit (SDK), by using implementation of the Hue SDK
most interactions with the lighting that would require the use of the
RESTful API are now able to be accessed using Java methods. The Hue SDK
exists alongside a very basic example Android application that allows
the connection to the Hue bridge used for interfacing with the lights
and a button to randomly adjust the hue values of the lights connected
to the bridge.

\subsubsection{Influence}\label{influence}

The implementation example of the bridge connection and the storage of
the API key required to authenticate a connected device/application
included was very useful for the initial set-up for interfacing with the
Hue lighting platform and as such a direct use of the RESTful API has
not been implemented within my application to interface with the Hue
lighting but instead the Hue SDK.

\subsection{Design Considerations}\label{design-considerations}

Through my research I have found that it has been shown bright lighting,
blue light in particular can affect the circadian rhythm of the body
which is used to regulate sleep patterns \cite{oh2015analysis}. There
are applications used to reduce the amount of blue light displayed in
the evening and through the night, applications such as flux
\parencite{flux} and features included within mobile operating systems
such as iOS \parencite{iosNightShift} and Android platforms
\parencite{samsungBlueFilter}.

\subsubsection{Influence}\label{influence-1}

My application will often be used within a low-light or dark scenario
and as such I will design the application to use a dark colour scheme.

A light colour scheme could be quite energising to the user and could
potentially make them more alert which is not ideal before going to
sleep and could extend the duration it takes to fall asleep, resulting
in restlessness and a poor nights sleeps.

\subsection{Human Perception of Light}\label{human-perception-of-light}

Many human senses are based on a logarithmic scale, this is to say we
are far more able to distinguish changes in light or sound in the lower
band of the senses compared to higher, as such a small increase in
volume of a whisper will be a more distinguishable change in volume than
two jet engines roaring and increasing by their volume by a small
amount.

The same applies to sight, it is more important to distinguish details
in low light such as that from the moon compared to the light change of
daylight at varying times of the day. We do this to normalise our senses
to best suit our environment.

This kind of stimulus perception is defined as the just-noticeable
difference (JND). Firs summarised by Ernst Weber in 1834 his equation
was called Weber's Law and simply stated that response intensity
increases as stimulus intensity increases
\parencite[p. 1613-1615]{salkind2010encyclopedia}. Further refined by
Gustav Fechner who proposed the use of a constant to provide a curve to
the stimulus/perception relationship. Fechners' law was a much better
fit, however some stimulus did not fit well, such as that of electric
shock.

Most recently in the 60's an American psychologist S. S. Stevens
produced a formula that worked for all forms of stimulus, even for
electric shocks \parencite{stevens1957psychophysical}. He proposed an
exponential function raising the data to a power rather than using a
simple constant. This essentially stated that to get a linear increase
in perception of various stimuli, the stimulus would need to increase in
an exponential form.

\subsubsection{Influence}\label{influence-2}

Due to this perception of light I feel increasing the brightness of the
lighting in the room in a logarithmic fashion will produce a perceptibly
more natural and linear increase in brightness over increasing the value
directly from 0 to the maximum value.
