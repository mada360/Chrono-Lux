\section{Project Proposal - Smart
alarm}\label{project-proposal---smart-alarm}

\subsection{The Problem}\label{the-problem}

As daily lives get more hectic and waking up is last thing most people
would like to do, I would like to develop an app that would try to make
both more manageable.

Currently there are hundreds of alarm apps available on `the Google Play
store' and with Google now many people have access to their own personal
digital assistant, but neither of these have been combined.

Alarm apps that can connect to `smart-bulbs' is even less common with
only a handful able to use their functionality.

\subsection{Introduction}\label{introduction}

My idea would be to create an alarm app that would allow users to have
their day in the palm of their hands for when they wake up.

\subsubsection{Problem}\label{problem}

\begin{itemize}
\item
  There is no alarm app that also utilises calendars, reminders or the
  weather.
\item
  Few apps can interact with smart-bulbs
\end{itemize}

\subsubsection{Solution}\label{solution}

\begin{itemize}
\item
  I will develop an android app that will be able to combine a users
  schedule and weather into a single app and allow for visual and audio
  output of information the user would like.
\item
  The Alarm app will also be able to connect to smart-bulbs allowing for
  an artificial sunrise believed to help in a more graceful awakening.
\end{itemize}

\subsection{How?}\label{how}

I have stated what the problem is and what I intend to do about it, but
how will I bring this application to fruition?

\subsubsection{Platform}\label{platform}

I will be developing primarily for the Android platform. This is due to
multiple factors;

\begin{enumerate}
\def\labelenumi{\arabic{enumi})}
\item
  Android is currently the most widely adopted platform globally.
\item
  Development for Android is free.
\item
  I possess/have access to multiple different Android devices allowing
  for a broader range of testing on physical devices.
\end{enumerate}

\subsubsection{Implementation}\label{implementation}

By using Android studio the development suite based on Ideas Intellij
development platform I will be able to write the application in Java
utilising my previous education over the previous years of University.

The Android studio provides the Android libraries, virtual device
emulation and easy live device usage making development simpler.

\subsubsection{Challenges}\label{challenges}

There will be many challenges to face while developing this app such as;

\begin{itemize}
\item
  Utilising the smart-bulb API(s).
\item
  The fractured nature of Android could make development more difficult.
\item
  Although Android is based on Java, it has many of its own libraries
  that I will need to learn.
\item
  The app will involve requesting data from other services (calendar)
  that the user may not provide and I will need to handle gracefully.
\end{itemize}

\subsubsection{Deliverables}\label{deliverables}

\begin{itemize}
\tightlist
\item
  Daily alarm with schedules, recurrence, customisable alarm tone.
\item
  Able to display appointments and weather in an attractive and simple
  fashion.
\item
  On disabling the alarm to read out schedule and offer suggestions. For
  example, if an umbrella/coat would be needed in the event of rain, or
  to leave within a set time to arrive at an appointment where the
  location has been entered.
\item
  To gradually increase the brightness of a smart-bulb to simulate the
  effect of sunrise. The app must also be able to control the light(s)
  in other ways such as turning them off at night/in the morning.
\end{itemize}

\subsubsection{Alternatives and
competition}\label{alternatives-and-competition}

For what my app intends to do there is one very good application
available currently in the Play store, this app is called
\href{https://play.google.com/store/apps/details?id=com.urbandroid.sleep}{`Sleep
as Android'}. It contains all of the functionality of my app including
many more and is designed in a way that fits the Android design
guidelines well.

The issue I find is there is too much and many features are either not
very effective such as the feature to track your sleep by placing your
phone on the bed and leaving it running over night, this doesn't work
for most mattresses and only older spring style. There is also the
functionality to set QR codes and place them around your house, such as
in the bathroom and to turn off the alarm you must get up and scan the
QR code; this sounds like quite a good idea but would leave rather ugly
QR codes around the house and for families or couples could be very
annoying.

Sleep as Android also doesn't include the calendar and to-do list
integration that I intend to include, nor does it have weather forecasts
or text to speech functionality.

This app is very good example of how I could design my app and to not
only fit the ecosystem I am developing for but also make it suitable for
it's purpose such as the dark background/theme which would be suitable
for night-time and low light use.

\subsection{Conclusion}\label{conclusion}

The problem I have found with the current alarm apps available is their
lack of time management or how helpful they are to the user. By
developing an alarm app that can provide integration with calendars and
weather it will provide relevant and useful information first thing in
the morning so the user can know everything they need for the day ahead.

My app should be very useful to many users, including myself and with
plenty of customisation should make for a more pleasant morning
experience.
